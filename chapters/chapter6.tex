\chapter{总结与展望}

\section{工作总结}

分心驾驶导致的交通事故率一直是居高不下,分心行为的研究一直持续不断,从最开始的统计分析分心的行为,到机器学习的特征提取分析是一个质的转变,标志着研究工作开始转变为一个新的阶段,可以在实际工程中使用,帮助驾驶者提升注意力,降低事故发生率。然而,机器学习方法在长时间使用过程中,特征提取越来越发杂,加之算法的不可靠性,使得最终机器学习会丧失掉其准确预测。

本文是依据分心驾驶行为检测发展的历程,立足于当前最新的视觉发展技术,结合于工程使用场景,提出了一种端到端的分析检测算法,力争于在工程使用中,提供高准确率的同时,也能有很好的鲁棒性。

本文的主要工作内容总结如下:

(1)	由于图像采集设备的固有缺陷,采集到的图像信息不可避免的夹杂着噪声,根据对图像噪声的统计研究发现,高斯噪声的占比是最高的。统计分析了图像信息中不同噪声强度对于图像信息的影响,及施加不同噪声强度,图像识别准确率影响的变化情况。确定了去噪处理的必要性,经过分析,最终确定了图像噪声强度检测的阈值,用于判定图像是否需要进行去噪处理。其次,针对图像中高斯噪声占比重的情况,选择了高斯滤波去噪技术,对图像进行去噪处理。

(2)	使用卷积神经网络提取图像特征信息,并分析了卷积神经网络中的代表作VGG网络与ResNet残差网络对于图像特征提取的效率问题,得到残差卷积神经网络提取效率目前是高效的一类网络,并且其所附含的残差结构,又使得之后的Vision Transformer网络结构的对于卷积提取的特征图,在进行Vision Transformer训练时可以保留更多的特征信息。并且,在二者进行融合时,针对卷积采集到的信息过载的问题,提出了一个针对Vision Transformer结构的序列图,及ResNet特征提取序列图像块的局部均值池化技术,使得Vision Transformer在保持足够特征信息的图像序列快快速训练,降低了模型的计算量,同样保持了较高的精度。在整体的测试过程中,从Vision Transformer对于特征效果的若提取能力的缺点出发,使用到ResNet对特征的高效提取能力相补充。另一方面,Vision Transformer对于图像远距离特征的联系关系,使得卷积神经网路一直以来局部特征联系性的弊端问题得到解决机会,二者的相互结合更能符合视觉网络的理想处理效能。

(3)	针对训练过程中,分类数据集存在的数据集不平衡问题,即多数样本的类别对少数类别样本的角度域空间的挤压问题,使得模型在不断的迭代中,从根本上使得出现多数样本类别预测概率大于少数样本类别概率。针对此问题,从分类问题最常使用Softmax监督函数出发,结合分类问题的理想模型,即“深度特征的区分度越高越好,不同类别样本特征之间的距离(差异性)越大越好,同类别样本特征之间的距离尽可能的小”,分析了另外几种常见的克服数据集不平衡方法的出发点。最后,借鉴了中心损失函数的类间聚集思想,以类别特征夹角固定思维出发,最类别中心的聚集采用了自适应的方法,提出了孤立中心损失函数用于分类任务的监督学习。在实验中验证了其优良的特性,符合设计的目的。

结合以上介绍的工作,设计了一个工程模拟系统,用于检测分心驾驶行为树算法的有效性。主要涵盖了图像噪声强度估计、高斯滤波去噪、RViT模型鉴别驾驶行等等。工程系统模拟达到了驾驶行为的鉴别,并保持了高准确度,验证了算法的可行性。


\section{工作展望}

本文是从分心驾驶行为的预训练角度出发,针对数据集存在数据不平衡的客观问题,立足于端到端的模块化设计方法,提出了孤立中心损失函数来克服相应的问题;并在分析了模型处理图像特征缺少远距离特征相关性的问题后,提出了RViT模型,来克服此问题。就本文目前的研究情况而言,应从以下几方面进行研究:

(1)	深入研究RViT的模型,针对RViT模型难训练的问题,进行优化,解决其中参数设置难度大的问题。并在特征计算复杂性等方向着手,设计更加方便的端到端模型。

(2)	针对监督学习的损失函数问题,孤立中心损失函数在大规模多类别数据集中使用时出现的,样本类别过多,损失函数特征惩罚难度过大的问题,研究分析解决思路。另一方面,只因深度学习是基于大规模数据,并非所有的样本类别只是简单失衡,也可能出现数据类别样本数量断崖式差距,并且在不断的测试中,出现特征区分度并未达到相应的等分效果,应该引起重视。
